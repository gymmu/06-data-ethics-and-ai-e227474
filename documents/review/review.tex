\documentclass{article}

\usepackage[ngerman]{babel}
\usepackage[utf8]{inputenc}
\usepackage[T1]{fontenc}
\usepackage{hyperref}
\usepackage{csquotes}

\usepackage[
    backend=biber,
    style=apa,
    sortlocale=de_DE,
    natbib=true,
    url=false,
    doi=false,
    sortcites=true,
    sorting=nyt,
    isbn=false,
    hyperref=true,
    backref=false,
    giveninits=false,
    eprint=false]{biblatex}
\addbibresource{../references/bibliography.bib}

\title{Review des Papers "Ethik im Umgang mit Daten" von Ruben Nussbaumer}
\author{Nikolay Voropayev}
\date{\today}

\begin{document}
\maketitle

\abstract{
    Dies ist ein Review der Arbeit zum Thema Ethik im Umgang mit Daten von Ruben Nussbaumer.
}
\newpage

\section{Korrekturen}
Im erstem Satz der Arbeit gibt es einene Grammatischen Fehler, anstelle eines Fragezeichens steht dort ein Punkt.
\newline
\newline
2.2 hat teilweise Missinformation, da in 2.3 nicht das Problem erwaehnt wird, das jegliche \enquote{gerechte} Algorythmen anhand der Daten, welche sie bekommen ungerecht werden, das wird als \enquote{Algorythmic bias} bezeichnet und kann schwerwiegende Folgen haben. Es gibt Beweise von Faellen in denen das zu Probemen gefuerht hat wie zum Beispiel KI Menschen mit Dunkler haut viel oefter als Kriminell bezeichnet oder Frauen als minderwertige Arbeiter von Amazazons Resume-scanning-tool angesehen worden. So sehr sogar, dass alle Resumes mit Daten, die auf eine Frau hinweisen wuerden, verworfen wurden. 
\newline
\citep{algorythmic-bias}
\newline
\citep{ai-social-credit-scores}
\newline
\citep{criminal-screening-ai-bias}
\newline
\citep{amazon-hiring-ai-bias}
\newline
\newline
In 2.4 werden Lizensen vorgeschlagen, aber nicht wer die Lizensen geben sollte und unter welchen umstaenden, es gibt schon jetzt Versuche von Big-Tech-AI Firmen dies zu tun, jedoch schlagen diese vor, es selber zu tun. Also wuerde es eine Konzentration von Macht geben. Anstelle Lizensen auszugeben, was durch korrupte politiker zu einer Uebermacht fuer Big-Tech Firmen enden wird, sollte das Benutzen von KI fuer illegale Zwecke verboten werden, was es schon ist. Oder die Lizensen muessen an alle verteilbar sein, was dann den ganzen Sinn der Lizensen zerstoeren wuerde. Eine einzelperson, welche einen Chatbot zum programmieren laufen lassen will, sollte nicht vom Staat staerker verfolgt werden, als eine Millidarden- oder gar Trilliardenfirma. Entweder ist KI fuer alle zugaenglich oder es ist fuer alle verboten. Genauso wie alle Steuern zahlen.

\section{Verbesserungen}
Die Einleitung ist etwas verwirrend, in 1.1 wird die Definition einer KI erklaert, aber wie sie funktioniert erst in 1.2.3. Ich wuerde vorschlagen, 1.1 auf \enquote{Was versteht man unter einer KI?} umbenennen. 
\newline
\newline
Es wird nicht erklaert, was Daten sind, es waere besser falls es \verb|\citep{}| - Quellen gaebe, auch nicht nur fuer diese Quelle. Zum Beispiel fuer 1.2.2 gibt es gar keine Quellen.
\newline
\newline
Die Abbildung 1.1 sollte unter 1.2.3 verschoben werden.

\section{Kommentare}
Die erklaerung in 1.2.3 ist sehr gut. Sie ist nich zu komplex, enifach und versteandlich formuliert und gibt gute Beispiele aus dem echtem Leben.

\printbibliography

\end{document}
