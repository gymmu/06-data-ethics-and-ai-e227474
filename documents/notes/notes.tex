\documentclass{article}

\usepackage[ngerman]{babel}
\usepackage[utf8]{inputenc}
\usepackage[T1]{fontenc}
\usepackage{hyperref}
\usepackage{csquotes}

\usepackage[
    backend=biber,
    style=apa,
    sortlocale=de_DE,
    natbib=true,
    url=false,
    doi=false,
    sortcites=true,
    sorting=nyt,
    isbn=false,
    hyperref=true,
    backref=false,
    giveninits=false,
    eprint=false]{biblatex}
\addbibresource{../references/bibliography.bib}

\title{Notizen zum Projekt Data Ethics}
\author{Nikolay Voropayev}
\date{\today}

\begin{document}
\maketitle

\abstract{
    Dieses Dokument ist eine Sammlung von Notizen zu dem Projekt. Die Struktur innerhalb des
    Projektes ist gleich ausgelegt wie in der Hauptarbeit, somit kann hier einfach geschrieben
    werden, und die Teile die man verwenden möchte, kann man direkt in die Hauptdatei ziehen.
}

\tableofcontents

\section{Ideen}
Google fotos vorfall (man wird als crimineller von KI bestimmt)

KI wird von richtern verwendet

KI kann Daten besser verarbeiten und ist deswegen fuer die Privatsphaere sehr schlecht

KI kann verwendet werden, um mit hilfe von gesichtserkennung profile von menschen erstellen, welche sie nicht loswerden koennen, auch wenn diese personen komplett das benutzen von technologie aufgeben.

KI wird verwendet um werbung gezielt anzuzeigen.

KI ist rassistisch
\section{Künstliche Intelligenz}
\label{sec:ai}

In diesem Abschnitt sind meine Notizen zu künstlicher Intelligenz zu finden.

Künstliche Intelligenz ist ein Teilgebiet der Informatik und beschäftigt sich mit maschinellem Lernen \citep{ai-wikipedia}.

AI\citep{ai-ibm}

ML\citep{machine-learning-ibm}

DL\citep{deep-learning-ibm}

THO\citep{Ai-is-running-out-of-water}

\printbibliography

\end{document}
