\documentclass{report}

\usepackage[ngerman]{babel}
\usepackage[utf8]{inputenc}
\usepackage[T1]{fontenc}
\usepackage{hyperref}
\usepackage{csquotes}
\usepackage[a4paper]{geometry}

\usepackage[
    backend=biber,
    style=apa,
    sortlocale=de_DE,
    natbib=true,
    url=false,
    doi=false,
    sortcites=true,
    sorting=nyt,
    isbn=false,
    hyperref=true,
    backref=false,
    giveninits=false,
    eprint=false]{biblatex}
\addbibresource{../references/bibliography.bib}


\title{Data Ethics and AI}
\author{Nikolay Voropayev}
\date{\today}


\begin{document}

\maketitle

\abstract{
    In diesem Dokument wird grob erklaert wie KI funktioniert, es werden die Gefahren von KI analysiert, logisch behandelt und schlussfolgerungen gezogen, welchen beweisen sollen, dass 
    \begin{enumerate}
        \item KI is nicht wirklich intelligent
        \item KI wird uns nicht ausloeschen wie in der Terminator-Franchise. 
        
        \url{https://en.wikipedia.org/wiki/The_Terminator}
        \item KI soll nicht nur in den haenden von Big-Tech Firmen ueberlassen werden, sondern sollte open-source gehalten werden.
        
        \url{https://en.wikipedia.org/wiki/Open-source_software}
        \item Datenschutz im zusammenhang mit KI is umsomehr wichtig als normalerweise.
    \end{enumerate}
}

\tableofcontents

\chapter{Einleitung}

Hier kommt die Einführung. Der Text hier sollte eigentlich noch viel länger sein, so das hier nicht so merkwürdige Umbrüche entstehen.

Ich kann weitere Kapitel auch importieren.

\input{chap_methode.tex}

\section{Etwas mit Quellen}

Etwas mit Änderung hier am Ende.

Wenn ich eine Quelle zitieren möchte, kann ich das ganze einfach am Ende des Satzes machen \citep{example}. Oder wie \citet{example} sagt, auch mitten im Text.

\printbibliography

\end{document}
